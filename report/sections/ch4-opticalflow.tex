\clearpage
\section{Optical flow sensor}

An optical flow sensor uses a camera to detect motion on a the $x$ and $y$ plane. It looks for movement by finding patterns on the ground to be able to determine the direction it is travelling in and how far the sensor has moved. This is vital information for SLAM mapping, since it is necessary to know where the mapping device is placed compared to previous measurements, this means that the optical flow sensor is used for finding the reference points. The sensor should be mounted quite high above ground on the rover to get a bigger view of the ground area, and is faced downwards to the ground\cite{opti_using}.

The optical flow sensor works in the same manner as a computer mouse and returns information to the rover. What is important is the surface that the optical flow is looking at. If it is reflective or has many differences that the camera can use for reference, the readings will be incorrect. It is therefore good to have a backup system to ensure that the rover receives a correct reference point, like encoders on the wheels so the program knows that the rover is moving.

We used the CJMCU-110 Optical flow V1.0 from 3Drobotics. It uses the ADNS-3080 optical chip. It can take both 3V and 5V. It uses the Serial Peripheral Interface (SPI) bus to communicate with the master\cite{opti_datasheet}.

\begin{figure}[H]
	\centering
	\includegraphics[width=.3\linewidth]{images/optical.jpg}
	\caption{The optical flow sensor from 3D Robotics.}
\end{figure}

During the development phase, we found out that no optical flow sensor was required for the SLAM package of our choice for ROS.