\pdfbookmark[0]{English title page}{label:titlepage_en}
\aautitlepage{%
  \englishprojectinfo{
    Mapping and navigation\\ of unknown terrain
  }{%
    Scientific Theme %theme
  }{%
    Spring Semester 2015 %project period
  }{%
    H103 % project group
  }{%
    %list of group members
	Antal János Monori\\
	Emil Már Einarsson\\
	Gustavo Smidth Buschle\\
	Thomas Thuesen Enevoldsen
  }{%
    %list of supervisors
	Akbar Hussain\\
	Torben Rosenørn
  }{%
    3 % number of printed copies
  }{%
    \today % date of completion
  }%
}{%department and address
  \textbf{School of Information and Communication Technology}\\
  Niels Bohrs Vej 8\\
  DK-6700 Esbjerg\\
  \href{http://sict.aau.dk}{http://sict.aau.dk}
}{% the abstract
	This project looks at the future needs for autonomous exploration of unknown environments and terrain. In the coming years humanity will push for more planetary and deep-sea exploration, and during this autonomous vehicles with capabilities of mapping unknown environments will play a major role. Maps of unknown planets or unmapped ocean bed will help prevent hazardous expeditions and will also gather data for researchers to process before visiting the unknown terrain with a larger scale expedition. Using a laser, ultrasonic sensors and a chassis, a low intelligence autonomous rover was created. It is capable of creating 2D maps using Hector SLAM on ROS with readings from the laser. The ultrasonic sensors were used to create a close proximity detection system, which helped the rover avoid obstacles it encountered in the unknown environments.
}