%resources
%    gas giants
%        helium is important
%    earthlike
%        earthlike planets are often composed of silicate and metals.
%
%
%Environment
%    atmosphere
%        Wind speeds
%        pressure (mercury none - venus 92a - gas giants a fuckton - http://www.compoundchem%.com/2014/07/25/planetatmospheres/)
%
%    terrain
%        rocky/sandy/none
%
%    temperature (Solar system 500C to -250C - https://solarsystem.nasa.gov/multimedia/display.cfm?IM_ID=169)
%
%conclusion
%    drones
%    Tracks/wheels
%        tracks are heavier than wheels, but better for rough terrain

\subsection Resources
For the purpose of exploration and resource gathering, planets can be split into two categories. The gassy giants, and the earthlike/rocky planets[ref?].
There is very little reason to explore a gassy giant, as they smoothly transition between gas, lquid, and solid the deeper you look[citation needed]. This means that it is extremely difficult to colosize such a planet, or build anything on it for that matter. However, they(solar system) have a relatively high helium content, which is a valuable gas for research[ref?]. 
In the other hand, earthlike planets have a larger variety of resources, and it is easier to build on them. These planets are often composed largely of metals and silicon[citation needed], which is what gives them their rocky and sandy surface.


\subsection{Environment}
\subsubsection{Atmosphere}
There are quite a few planets with a negligible atmosphere. In these planets sound based sensors cannot work, but light based sensors might even work better.
Planets with an atmosphere can be much harder to cope with. The biggest aspect of a planets atmosphere is the pressure at surface level.

\subsubsection{Terrain}
Earthlike planets don't have a very big variation in terrain, as most of them do not have any liquid substances that could change the terrain.


\subsection{Conclusion}

